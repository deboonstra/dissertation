\documentclass{article}
\usepackage[utf8]{inputenc}
% for math
\usepackage{amsmath}
\usepackage{amssymb}

% for hyperlinks
\usepackage{hyperref}
% for captions
\usepackage{caption}
% strict table placement
\usepackage{float}
% Setting margins
\usepackage{geometry}
\geometry{margin = 1in}
% colors
\usepackage{xcolor}

%from https://github.com/deboonstra/boonstra-latex.git
\input{commands}

\title{Notation}
\author{D. Erik Boonstra, MS}
\date{Summer 2023}

\begin{document}
\maketitle

\noindent The current version of the notation for this project is based on Pan 2001a with some modifications made by D. Erik Boonstra and Joseph E. Cavanuagh. 

\section*{Data generation}
\noindent Suppose we have a random sample of observations from $N$ individuals. For each individual $i$, we have
\begin{itemize}
  \item $\y_{i} = \begin{pmatrix}
    y_{i1} & \ldots & y_{in_{i}}
  \end{pmatrix}\Tr$,
  \item covariates $\X_{i} = \begin{pmatrix}
    \x_{i1}\Tr & \ldots & \x_{in_{i}}\Tr
  \end{pmatrix}\Tr$, where $\x_{ij}$ is a $p-$dimensional vector,
  \item conditional on the covariates the observations of $\y_{i}$ are correlated but $\y_{i}$ and $\y_{k}$ are independent for all $i \neq k$ ,
  \item $\cD = \left\{(\y_{1}, \X_{1}), \ldots, (\y_{N}, \X_{N})\right\}$, and
  \item $g(\bm_{i}) = \X_{i}\bb$,
\end{itemize}
where $\Ex(\y_{i}\mid\X_{i}) = \bm_{i}$, $g(\cdot)$ is a specified link function, and $\bb = \begin{pmatrix}
  \beta_{1} & \ldots & \beta_{p}
\end{pmatrix}\Tr$ is a vector of unknown regression parameters to be estimated.

\section*{Generalized estimating equations}
\noindent For \textit{generalized estimating equations} (GEEs), the estimates for $\bb$ are obtained by using the following estimating equations (Liang and Zeger, 1986)
\begin{equation}
  \label{eqn:gee}
  S(\bb; \R, \cD) = \sum_{i = 1}^{N}\D_{i}\Tr\V_{i}^{-1}(\y_{i} - \bm_{i}) = 0,
\end{equation}
where $\D_{i} = \D_{i}(\bb) = \partial\bm_{i}(\bb)/\partial\bb\Tr$ and $\V_{i}$ is the working covariance matrix of $\y_{i}$. This covariance matrix, $\V_{i}$, can be expressed as
\begin{equation*}
  \V_{i} = \A_{i}^{1/2}\R_{i}(\ba)\A_{i}^{1/2},
\end{equation*}
with $\R = \R_{i}(\ba)$ being the working correlation matrix and $\A_{i}$ is a diagonal matrix of marginal variance, $\Var(y_{ij}) = \phi\nu(\mu_{ij})$, where $\phi$ is the dispersion parameter and $\nu(\cdot)$ is mean-to-variance relationship defined by link function, $g(\cdot)$. GEEs are based on the quasi-likelihood. So, for simplicity, suppose we have a scalar response variable, $y$. Then, the (log) quasi-likelihood function defined by McCullagh and Nelder, 1989 is
\begin{equation}
  \label{eqn:qfn}
  Q(\mu, \phi; y) = \int_{y}^{\mu} \frac{y - t}{\phi\nu(t)}dt,
\end{equation}
where $\mu = \Ex(y)$ and $\Var(y) = \phi\nu(\mu).$ With a $1 \times p$ covariate $\x$ and a specified regression model
\begin{equation*}
  \Ex(y) = \mu = g^{-1}(\x\bb),
\end{equation*}
the quasi-likelihood may be written as
\begin{align*}
  Q(\mu, \phi; y) &= Q(g^{-1}(\x\bb), \phi; y) \\
  &= Q(\bb, \phi; (y, \x)).
\end{align*}
Now if the working independence model, $\R = \I$ is used, the working assumption is that the paired observations $(y_{ij}, \X_{ij})$ in $\cD$ are independent. Thus,
\begin{equation}
  \label{eqn:qbeta}
  Q(\bb, \phi; \I, \cD) = \sum_{i = 1}^{N}\sum_{j = 1}^{n_{i}}Q\left[\bb, \phi; (y_{ij}, \X_{ij})\right],
\end{equation}
and it is easy to see that $S(\bb; \I, \cD)$ is equivalent to $\partial Q(\bb, \phi; \I, \cD)/\partial\bb$. The likelihood equivalent to $Q(\bb, \phi; \I, \cD)$ would be the log-likelihood, $\ell(\bb, \phi; \cD),$ function. For the exponential dispersion family, $Q(\bb, \phi; \I, \cD)$ and $\ell(\bb, \phi; \cD)$ are equivalent up to a constant, $c$. Assuming $\phi$ is known or estimated by another set of estimating equations, the estimate for $\bb$ is $\bb(\R)$ using the estimating equations in \ref{eqn:gee}. Additionally, the $\cov(\bbh)$ has a consistent robust or sandwich based estimator, $\hat{\V}_{r}$; and, the empirical or model-based estimator is
\begin{equation*}
  \hat{\bO}_{r} = \frac{-\partial^{2}Q(\bb; \R, \cD)}{\partial\bb\partial\bb\Tr}\bigg|_{\bb = \bbh(\R)}.
\end{equation*}

\section*{Data transformation}
With the robust estimator of $\cov(\bbh)$ now being defined, suppose $\y_{i}^{*}$ and $\X_{i}^{*}$ are $\y_{i}$ and $\X_{i}$ transformed by $\R$ such that
\begin{align*}
  \y_{i}^{*} &= \hat{\V}_{r}^{-1/2}\y_{i} \\
  \X_{i}^{*} &= \hat{\V}_{r}^{-1/2}\X_{i},
\end{align*}
and $\cD^{*} = \left\{(\y_{1}^{*}, \X_{1}^{*}), \ldots, (\y_{N}^{*}, \X_{N}^{*})\right\}.$

\pagebreak

\section*{Information criterions}
The table below is the current collection of the information criterions proposed and investigated. Using the notation defined previously and letting GOF abbreviate goodness-of-fit term, we have the following.

\begin{table}[H]
  \centering
  \begin{tabular}{lcccl}
    \hline
    \hline
    Name & Abbreviation & GOF & Penalty & Notes \\
    \hline
    & & & & \\
    "An" information & AIC & $-2\ell(\bbh, \cD)$ & $2p$ & \\
    criteria & & & \\
    Bayesian information & BIC & $-2\ell(\bbh, \cD)$ & $\log(N)p$ & \\
    criterion & & & & \\
    Quasi-likelihood & QIC(R) & $-2Q(\bbh(\R), \I, \cD)$ & $2tr(\hat{\bO}_{\I}\hat{\V}_{r})$ & \\
    under independence & & & & \\
    model criterion & & & & \\
    & QICu(R) & $-2Q(\bbh(\R), \I, \cD)$ & $2p$ & $tr(\hat{\bO}_{\I}\hat{\V}_{r}) = p \text{ if } \hat{\bO}_{\I} \approxeq \hat{\V}_{r} $ \\
    & BQICu(R) & $-2Q(\bbh(\R), \I, \cD)$ & $\log(N)p$ & Completely made up by DEB \\
    Correlation information & CIC(R) & & $2tr(\hat{\bO}_{\I}\hat{\V}_{r})$ & \\
    criteria & & & & \\
    \hline
    & & & & \\
    & $\text{AIC}^{*}$ & $-2\ell(\bbh^{*}, \cD^{*})$ & $2p$ & \\
    & $\text{IC}^{*}(\text{R})$ & $-2\ell(\bbh^{*}, \cD^{*})$ & $2tr(\hat{\bO}_{\I}\hat{\V}_{r})$ & \\
    & $\text{QIC}^{*}(\text{I})$ & $-2Q(\bbh^{*}(\I), \I, \cD^{*})$ & $2tr(\hat{\bO}_{\I}^{*}\hat{\V}_{r}^{*})$ & \\
    & $\text{QIC}^{*}(\text{R})$ & $-2Q(\bbh^{*}(\I), \I, \cD^{*})$ & $2tr(\hat{\bO}_{\I}\hat{\V}_{r})$ & Penalty is based on $\cD$ not $\cD^{*}$\\
    & $\text{QICu}^{*}(\text{I})$ & $-2Q(\bbh^{*}(\I), \I, \cD^{*})$ & $2p$ & \\
    & $\text{BQICu}^{*}(\text{I})$ & $-2Q(\bbh^{*}(\I), \I, \cD^{*})$ & $\log(N)p$ & \\
    & & & & \\
    \hline
    \hline
  \end{tabular}
\end{table}
\noindent Through simulations it was observed that $-2\ell(\bbh^{*}, \cD^{*}) = -2Q(\bbh^{*}(\I), \I, \cD^{*}) + c$. 
\end{document}